%	-------------------------------------------------------------------------------
% 
%
%
%
%
%
%
%
%
%
%	-------------------------------------------------------------------------------

	\documentclass[12pt, a3paper, landscape, oneside]{book}
%	\documentclass[12pt, a4paper, oneside]{book}
%	\documentclass[12pt, a4paper, landscape, oneside]{book}

		% --------------------------------- 페이지 스타일 지정
		\usepackage{geometry}
%		\geometry{landscape=true	}
		\geometry{top 		=10em}
		\geometry{bottom		=10em}
		\geometry{left		=8em}
		\geometry{right		=8em}
		\geometry{headheight	=4em} % 머리말 설치 높이
		\geometry{headsep		=2em} % 머리말의 본문과의 띠우기 크기
		\geometry{footskip		=4em} % 꼬리말의 본문과의 띠우기 크기
% 		\geometry{showframe}
	
%		paperwidth 	= left + width + right (1)
%		paperheight 	= top + height + bottom (2)
%		width 		= textwidth (+ marginparsep + marginparwidth) (3)
%		height 		= textheight (+ headheight + headsep + footskip) (4)



		%	===================================================================
		%	package
		%	===================================================================
%			\usepackage[hangul]{kotex}				% 한글 사용
			\usepackage{kotex}					% 한글 사용
			\usepackage[unicode]{hyperref}			% 한글 하이퍼링크 사용

		% ------------------------------ 수학 수식
			\usepackage{amssymb,amsfonts,amsmath}	% 수학 수식 사용
			\usepackage{mathtools}				% amsmath 확장판



		% ------------------------------ 
			\usepackage{fix-cm}	
			\usepackage[english]{babel}


		% ------------------------------ table 
			\usepackage{longtable}			%
			\usepackage{tabularx}			%
			\usepackage{tabu}				%



		%	--------------------------------------------------------------------------------------- 
		% 	tritlesec package
		% 	
		% 	
		% 	------------------------------------------------------------------ section 스타일 지정
	
			\usepackage{titlesec}
		
		% 	----------------------------------------------------------------- section 글자 모양 설정
			\titleformat*{\section}					{\large\bfseries}
			\titleformat*{\subsection}				{\normalsize\bfseries}
			\titleformat*{\subsubsection}			{\normalsize\bfseries}
			\titleformat*{\paragraph}				{\normalsize\bfseries}
			\titleformat*{\subparagraph}				{\normalsize\bfseries}

			\titleformat{\section}   {\normalfont\large\bfseries}{\thesection}{4em}{}	
	
		% 	----------------------------------------------------------------- section 번호 설정
			\renewcommand{\thepart}				{\arabic{part}.}
			\renewcommand{\thesection}				{\arabic{section}.}
			\renewcommand{\thesubsection}			{\thesection\arabic{subsection}.}
			\renewcommand{\thesubsubsection}		{\thesubsection\arabic{subsubsection}}
			\renewcommand\theparagraph 			{$\blacksquare$ \hspace{3pt}}

		% 	----------------------------------------------------------------- section 페이지 나누기 설정
			\let\stdsection\section
			\renewcommand\section{\newpage\stdsection}



		%	--------------------------------------------------------------------------------------- 
		% 	\titlespacing*{commandi} {left} {before-sep} {after-sep} [right-sep]		
		% 	left
		%	before-sep		:  수직 전 간격
		% 	after-sep	 	:  수직으로 후 간격
		%	right-sep

			\titlespacing*{\section} 				{0em}{1.0em}{1.0em}
			\titlespacing*{\subsection}	  		{0ex}{1.0em}{1.0em}
			\titlespacing*{\subsubsection}			{0ex}{1.0em}{1.0em}
			\titlespacing*{\paragraph}			{0em}{1.5em}{1.0em}
			\titlespacing*{\subparagraph}			{4em}{1.0em}{1.0em}
	



		%	--------------------------------------------------------------------------------------- 
		% 	toc 설정  - table of contents
		% 	
		% 	
		% 	----------------------------------------------------------------  문서 기본 사항 설정
			\setcounter{secnumdepth}{5} 		% 문단 번호 깊이
			\setcounter{tocdepth}{3} 			% 문단 번호 깊이 - 목차 출력시 출력 범위

			\setlength{\parindent}{0cm} 		% 문서 들여 쓰기를 하지 않는다.


		%	--------------------------------------------------------------------------------------- 
		% 	mini toc 설정
		% 	
		% 	
		% 	--------------------------------------------------------- 장의 목차  minitoc package
			\usepackage{minitoc}

			\setcounter{minitocdepth}{1}    	%  Show until subsubsections in minitoc
%			\setlength{\mtcindent}{12pt} 	% default 24pt
			\setlength{\mtcindent}{24pt} 	% default 24pt

		% 	--------------------------------------------------------- part toc
		%	\setcounter{parttocdepth}{2} 	%  default
			\setcounter{parttocdepth}{0}
		%	\setlength{\ptcindent}{0em}		%  default  목차 내용 들여 쓰기
			\setlength{\ptcindent}{0em}         


		% 	--------------------------------------------------------- section toc

			\renewcommand{\ptcfont}{\normalsize\rm} 		%  default
			\renewcommand{\ptcCfont}{\normalsize\bf} 	%  default
			\renewcommand{\ptcSfont}{\normalsize\rm} 	%  default


		%	=======================================================================================
		% 	tocloft package
		% 	
		% 	------------------------------------------ 목차의 목차 번호와 목차 사이의 간격 조정
			\usepackage{tocloft}

		% 	------------------------------------------ 목차의 내어쓰기 즉 왼쪽 마진 설정
			\setlength{\cftsecindent}{2em}			%  section

		% 	------------------------------------------ 목차의 목차 번호와 목차 사이의 간격 조정
			\setlength{\cftsecnumwidth}{2em}		%  section




		%	=======================================================================================
		% 	tikz package
		% 	
		% 	--------------------------------- 	
			\usepackage{tikz}%
			\usetikzlibrary{arrows,positioning,shapes}
			\usetikzlibrary{mindmap}			




% ------------------------------------------------------------------------------
% Begin document (Content goes below)
% ------------------------------------------------------------------------------
	\begin{document}
			\dominitoc
			\doparttoc			




			\title{QGIS}
			\maketitle


			\tableofcontents 		% 목차 출력
%			\listoffigures 			% 그림 목차 출력
			\cleardoublepage
			\listoftables 			% 표 목차 출력



% 	==============================================================================  chapter  Flow chart
	\section{설치}
	\pagestyle{empty}

		%	----------------------------------------------- mind map
		% 	하향 5개
		%	----------------------------------------------- mind map
		\begin{center}
		\tikz[	mindmap,
%				minimum size=3cm,
				text width=6em, 
				concept color=black!80,
				level 1/.style={level distance=5.5cm,sibling angle=45},
				level 2/.style={level distance=5.0cm,sibling angle=45},
				level 3/.style={level distance=5.0cm,sibling angle=45},
				level 4/.style={level distance=5.0cm,sibling angle=45},
				concept/.append style={fill={none}} 
				]
				\node 	[concept] 				{프린트}
						[clockwise from=00, every concept/.style={minimum size=3cm}]
				child	{					node[concept] 	{LBP1400K}
						[clockwise from=45]
					child	{				node[concept] 	{설정}
						[clockwise from=00]
						child	{			node[concept] 	{네트워크}
							child	{		node[concept] 	{chungchul2.4G}} 
							} 
						}
					child	{				node[concept] 	{에러처리}
						[clockwise from=00]
						child	{			node[concept] 	{용지 크기가 맞지않음}
							child	{		node[concept] 	{온라인 누르면 해결}} 
							} 
						}
					child	{				node[concept] 	{토너}
						[clockwise from=00]
						child	{			node[concept] 	{용지 크기가 맞지않음}
							child	{		node[concept] 	{온라인 누르면 해결}} 
							} 
						}
					}
				child	{	node[concept] 	{이면지 출력용}
					}
				child	{	node[concept] 	{칼라 복사기}
					}
				child	{	node[concept] 	{송윤석 HP}
					};

		\end{center}

% 	==============================================================================  chapter  Flow chart
	\section{좌표계}
	\pagestyle{empty}

		\begin{tikzpicture}[	mindmap,
					    every node/.style=concept,
					    concept color=black!10,
					    grow cyclic,
%					    level 1/.append style={level distance=4.0cm,sibling angle=360},
					    level 1/.append style={level distance=6.0cm},
					    level 2/.append style={level distance=6.5cm}
					    level 3/.append style={level distance=6.5cm}
					    ]
		\draw	[step=1cm, black!40, very thin] (-16,-10) grid (16,10);
		\node 	[root concept] 					{좌표계}; % root


		\end{tikzpicture}




% 	==============================================================================  chapter  Flow chart
	\section{거리와 면적 계산하기}
	\pagestyle{empty}


% 	==============================================================================  chapter  Flow chart
	\section{투영법 확인하고 워프(재투영) 적용하기}
	\pagestyle{empty}



% 	==============================================================================  chapter  Flow chart
	\section{피처 선택하고 편집하기}
	\pagestyle{empty}



% 	==============================================================================  chapter  Flow chart
	\section{벡터 레이어 분석}
	\pagestyle{empty}


% 	==============================================================================  chapter  Flow chart
	\section{임상도, 항공사진, 지적도}
	\pagestyle{empty}




% 	==============================================================================  chapter  Flow chart
	\section{수치지질도}
	\pagestyle{empty}


% 	==============================================================================  chapter  Flow chart
	\section{피처 추가하기}
	\pagestyle{empty}


% 	==============================================================================  chapter  Flow chart
	\section{엑샐을 쉐이프 파일로 변환}
	\pagestyle{empty}


% 	==============================================================================  chapter  Flow chart
	\section{지형분석}
	\pagestyle{empty}


% 	==============================================================================  chapter  Flow chart
	\section{플러그인}
	\pagestyle{empty}


% 	==============================================================================  chapter  Flow chart
	\section{시진을 포인트 파일로 변환}
	\pagestyle{empty}








% ------------------------------------------------------------------------------
% End document
% ------------------------------------------------------------------------------
\end{document}



% =================================================================================================== Part 혼화 재료

