\documentclass[12pt]{article}

    \usepackage[ngerman]{babel}
    \usepackage[a4paper, text={16.5cm, 25.2cm}, centering]{geometry}
    \usepackage[sfdefault]{ClearSans}
    \usepackage[utf8]{inputenc}
    \setlength{\parskip}{1.2ex}
    \setlength{\parindent}{0em}

    \usepackage{tikz}
    \usetikzlibrary{mindmap}
    \usepackage[ampersand]{easylist}
    \begin{document}


    \begin{tikzpicture}[mindmap, grow cyclic,
		every node/.style=concept, 
		concept color=lime!80, 
		level 1/.append   style={level distance=5cm, sibling angle=120}, 
		level 2/.append style={level distance=3 cm, sibling angle=60}, 
		every annotation/.style={concept color=blue!20, text width={}, align=left}]

    \node{Lehramt}
child{node{Pädagogik}
        child{node{Steop}}
        child{node{Erziehen und Beraten}}
        child{node{Lehren und Lernen}}
        child{node{Schul\-architek\-tur}}
}
child{node{Sport}
    child{node{Physio\-logie}}
    child{node{Anatomie}}
    child{node{Inklusives}}
}
child[level distance=6cm]{node{Französisch}
    child{node{Sprach\-kurs}}
    child{node (n2) {Fach\-didaktik}}
    child{node{Sprach\-wissen\-schaft}} 
    child{node{Landes\-wissen\-schaft}}
    child{node{Medien\-wissen\-schaft}}
    child{node{Literatur\-wissen\-schaft}}
}
    ;
    \node [annotation, right] at (n2.east) {Französich unterrichten lernen \\ Wie gestaltet man Unterricht?} 
    ;
    \end{tikzpicture}

    \end{document}